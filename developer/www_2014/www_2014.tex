\documentclass{sig-alternate}

\begin{document}
\title{Accessibility to Scientific Material: The Case of Speaking Math}
\numberofauthors{2} %  in this sample file, there are a *total*

\author{
% You can go ahead and credit any number of authors here,
% e.g. one 'row of three' or two rows (consisting of one row of three
% and a second row of one, two or three).
%
% The command \alignauthor (no curly braces needed) should
% precede each author name, affiliation/snail-mail address and
% e-mail address. Additionally, tag each line of
% affiliation/address with \affaddr, and tag the
% e-mail address with \email.
%
% 1st. author
\alignauthor
Volker Sorge\titlenote{Work was done while author spent a sabbatical at Google,
  Inc., Mountain View, CA, USA.}\\
       \affaddr{School of Computer Science}\\
       \affaddr{The University of Birmingham, UK}\\
       \email{V.Sorge@cs.bham.ac.uk}
% 2nd. author
\alignauthor
Charles Chen, T.V. Raman, David Tseng\\
       \affaddr{Google, Inc.}\\
       \affaddr{Mountain View, CA, USA}\\
       \email{\{clchen|raman|dtseng\}@google.com}
}


\maketitle
\begin{abstract}
As we move away from the traditional methods of publishing and teaching to the
provision of web-based content more and more scientific and teaching material
becomes available online often in novel formats such as interactive diagrams or
simulations. Ensuring full accessibility to this material for users that rely
primarily on voice output from their computer becomes a challenging
problem. Already the text to speech translation of fairly conventional
scientific content like mathematical formulae, which contain rich structure and
for which a well defined markup language exists, is non-trivial.

We present our efforts of making the speech translation of
mathematical formulas a first class citizen in a general screen
reader. We demonstrate how we can deal with the collection of formats
that is presently used to render mathematics on the web and how we
extend the idea of letting users engage with content on different
levels of granularity to specialised content like maths.
\end{abstract}
\category{}{}{}
%A category including the fourth, optional field follows...
\category{}{}{}[]

\terms{}

\keywords{}

\section{Introduction}\label{sec:intro}
As we move away from the traditional methods of publishing and
teaching to the provision of web-based content more and more
scientific and teaching material becomes available online often in
novel formats such as interactive diagrams or simulations. Ensuring
full accessibility to this material for users that rely primarily on
voice output from their computer becomes a challenging
problem. Already the text to speech translation of fairly conventional
scientific content like mathematical formulae, which contain rich
structure and for which a well defined markup language exists, is
non-trivial.


In this paper we report on our effort to integrate the basic support
for voicing mathematical content on the web into the ChromeVox screen
reader. We shall present some of the main challenges to making maths
on the web accessible, where it is given in a variety of ways and
formats. We shall outline the rule based approach we have pursued and
how it fits with ChromeVox's philosophy to enable users to explore
content at different levels of granularity. It has led to an
implementation of a flexible speech rule engine that allows to
customise the reading experience along several axes and that also
provides an API for easy adaptation to specialised content by users
and web site authors.

\section{Mathematics on the Web}
\label{sec:math}

\subsection{MathML}
\label{sec:mathml}

\subsection{MathJax Rendering}
\label{sec:mathjax}

\subsection{Hidden Markup}
\label{sec:images}


\section{ChromeVox}
\label{sec:chromevox}

\section{Translating Mathematics}
\label{sec:translate}

\section{Exploring Mathematics}
\label{sec:explore}

\section{Alternative Representations}
\label{sec:alternative}

\section{Conclusions}
\label{sec:conc}


\bibliographystyle{plain}
\bibliography{www_2014}  

\end{document}
